
Title detection and Table of Content (ToC) extraction are two important tasks for Document Analysis, in particular in the context of digital libraries and scanned books.

We proposed two types of features for the Title Detection task, we used a naive Bayses classifier as a baseline and a decision tree (DT10). We showed that simple stylometric features (frequency of punctuation, numbers and capitalized letters) combined with visual characteristics  (bold, italic\dots) achieve better results than the best character n-gram approach (1-4 grams).
Although this system did not achieved state-of-the-art performances, the results shows that simple and easy-to-compute features can provide very reliable results.

Regarding the ToC Extraction task, we choose to extract the structure from
the ToC of the prospectuses.  We are pleased to see that are our
expectations are confirmed.  Our system obtains a good precision and
lower recall. For a next edition, we would like to focus on the extraction
of the structure from the whole document content.
